\documentclass[12pt]{article}
\usepackage{geometry}
\geometry{a4paper, margin=1in}
\usepackage{setspace}
\usepackage{amsmath}
\usepackage{parskip}
\usepackage{titling}
\usepackage{hyperref}
\usepackage{hanging}
\usepackage{url} % Ensure the url package is loaded
\renewcommand{\UrlFont}{\normalfont} % Change \UrlFont to the default font


\setlength{\droptitle}{-5em}

\title{Weight Loss Plan Critique: The Ketogenic Diet}
\author{\fontsize{11}{0}\selectfont{\underline{Name and Student Identification}: Arfaz Hossain, V00984826}\\
        \fontsize{11}{0}\selectfont{\underline{Date of Submission}: November 19, 2024}}

\date{}

\begin{document}

\maketitle
\doublespacing
\vspace{-2em}

The ketogenic diet, commonly known as the "keto diet," has gained significant popularity as a weight loss approach in recent years. This high-fat, very-low-carbohydrate diet aims to induce ketosis, a metabolic state where the body primarily uses fat for energy instead of carbohydrates (Ludwig, 2020). While proponents claim numerous benefits, including rapid weight loss and improved metabolic health, a critical examination of the scientific evidence reveals both promising aspects and significant concerns.\vspace{-2em}

\section*{Scientific Evidence and Health Implications}\vspace{-1em}
Research demonstrates that the ketogenic diet can lead to significant short-term weight loss. A comprehensive meta-analysis by Castellana et al.\ (2020) found that individuals following a ketogenic diet lost more weight compared to those on low-fat diets during the first year. The diet's effectiveness appears to be driven by multiple mechanisms, including reduced hunger due to ketosis, lower insulin levels, and decreased overall caloric intake (Gibson et al., 2015). However, long-term adherence remains a significant challenge, with studies tracking participants beyond one year showing high dropout rates and weight regain. Kirkpatrick et al.\ (2019) reported that only 45\% of participants maintained the diet after 12 months.

While the ketogenic diet shows promise for specific medical conditions, including epilepsy and certain metabolic disorders, its safety as a long-term weight loss strategy raises concerns. A systematic review by Crosby et al.\ (2021) highlighted several potential risks, including nutrient deficiencies, particularly in fiber, vitamins, and minerals typically obtained from restricted foods like fruits, whole grains, and legumes. The high intake of saturated fats characteristic of many ketogenic diets has also raised cardiovascular concerns, with studies showing increased LDL cholesterol levels in certain individuals (O'Neill \& Raggi, 2020).

Furthermore, research by Martinez-Gonzalez et al.\ (2019) suggests that severe carbohydrate restriction may negatively impact gut microbiome diversity, while Watanabe et al.\ (2020) documented common side effects including fatigue, headaches, and constipation during the initial adaptation period. These findings contrast significantly with the often-idealized portrayal of the diet in popular media and marketing materials.

\vspace{-2em}\section*{Recommendation}\vspace{-1em}
Based on the available scientific evidence, I would not recommend the ketogenic diet as a sustainable weight loss approach for the general population. While it may offer short-term benefits and specific therapeutic applications, the restrictive nature, potential health risks, and difficulty maintaining the diet make it unsuitable for long-term use. Instead, evidence-based approaches focusing on balanced nutrition and sustainable lifestyle changes provide safer and more reliable paths to weight management.\vspace{-2em}

\section*{References}\vspace{-1em}
\begin{hangparas}{.1in}{1}
Ludwig, D. S. (2020). The ketogenic diet: Evidence for optimism but high-quality research needed. \textit{Journal of Nutrition, 150}(6), 1354--1359. \url{https://doi.org/10.1093/jn/nxz308}

Castellana, M., Conte, E., Cignarelli, A., Perrini, S., Giustina, A., \& Giovanella, L. (2020). Efficacy and safety of very low calorie ketogenic diet (VLCKD) in patients with overweight and obesity: A systematic review and meta-analysis. \textit{Reviews in Endocrine and Metabolic Disorders, 21}(1), 5--16. \url{https://doi.org/10.1007/s11154-019-09514-y}

Gibson, A. A., Seimon, R. V., Lee, C. M., Ayre, J., Franklin, J., \& Markovic, T. P. (2015). Do ketogenic diets really suppress appetite? A systematic review and meta‐analysis. \textit{Obesity Reviews, 16}(1), 64--76. \url{https://doi.org/10.1111/obr.12230}

Kirkpatrick, C. F., Bolick, J. P., Kris-Etherton, P. M., Sikand, G., Aspry, K. E., \& Soffer, D. E. (2019). Review of current evidence and clinical recommendations on the effects of low-carbohydrate and very-low-carbohydrate diets. \textit{Journal of Clinical Lipidology, 13}(5), 689--711. \url{https://doi.org/10.1016/j.jacl.2019.08.003}

Crosby, L., Davis, B., Joshi, S., Jardine, M., Paul, J., Neola, M., \& Barnard, N. D. (2021). Ketogenic diets and chronic disease: Weighing the benefits against the risks. \textit{Frontiers in Nutrition, 8}, 702802. \url{https://doi.org/10.3389/fnut.2021.702802}

O'Neill, B., \& Raggi, P. (2020). The ketogenic diet: Pros and cons. \textit{Atherosclerosis, 292}(1), 119--126. \url{https://doi.org/10.1016/j.atherosclerosis.2019.11.021}

Martinez-Gonzalez, M. A., Gea, A., \& Ruiz-Canela, M. (2019). The Mediterranean diet and cardiovascular health. \textit{Circulation Research, 124}(5), 779--798. \url{https://doi.org/10.1161/CIRCRESAHA.118.313348}

Watanabe, M., Tozzi, R., Risi, R., Tuccinardi, D., \& Mariani, S. (2020). Beneficial effects of the ketogenic diet on nonalcoholic fatty liver disease. \textit{Obesity Reviews, 21}(4), e12977. \url{https://doi.org/10.1111/obr.13024}
\end{hangparas}

\end{document}