\documentclass[11pt]{article}
\usepackage{geometry}
\geometry{a4paper, margin=1.2in}
\usepackage{setspace}
\usepackage{amsmath}
\usepackage{parskip}
\usepackage{titling}
\usepackage{hanging}
\usepackage{url}
\renewcommand{\UrlFont}{\normalfont}

% Title spacing adjustment
\setlength{\droptitle}{-6em}

\title{Six-Week Nutritional Challenge Reflection}
\author{\fontsize{9.5}{0}\selectfont{\underline{Name and Student Identification}: Arfaz Hossain, V00984826}\\
        \fontsize{9.5}{0}\selectfont{\underline{Date of Submission}: November 28, 2024}}
\date{}

\begin{document}

\maketitle
\setstretch{1.2}
\vspace{-3em}

\section*{Challenge Overview}\vspace{-1em}
For my six-week nutritional challenge, I set two specific goals:
\begin{itemize}
    \item \textbf{Improve breakfast quality}: Replace my usual toast and eggs with a nutrient-dense smoothie containing a balanced blend of fruits, greens, nuts, and seeds.
    \item \textbf{Increase hydration}: Raise my daily water intake from 1.5L to 3.2L, aligned with the recommended intake for my weight (68kg) and activity level.
\end{itemize}
These goals were tracked using precise metrics and evaluated based on their impact on health and daily routine.

\section*{Rationale and Motivation}\vspace{-1em}
I aimed to address two primary areas: dietary diversity and hydration. My typical breakfast lacked variety, and I wanted to include more vitamins, fiber, and healthy fats to improve my energy and focus. Similarly, inadequate hydration had led to fatigue and reduced productivity, prompting my target of 3.2L of water daily. This amount was calculated based on my body weight and exercise regimen.

\section*{Challenge Methodology}\vspace{-1em}
\subsection*{Tracking Mechanisms}\vspace{-1.2em}
To monitor my progress, I used several key mechanisms. I tracked my water intake and smoothie consumption through my smartphone app (Microsoft To Do). I set up visual cues by pre-portioning my smoothie ingredients in my fridge and keeping my water bottles within reach. In my weekly reviews through the To Do application, I logged my observations including my energy levels, digestion, and skin health.

\subsection*{Hydration Strategy}\vspace{-1.2em}
To consistently consume 3.2L of water, I followed a segmented schedule:
\begin{itemize}
    \item \textbf{Morning (7:00 AM - 12:00 PM)}: 1L, spread over several hours  
    \item \textbf{Afternoon (12:00 PM - 5:00 PM)}: 1.2L, paced evenly throughout  
    \item \textbf{Evening (5:00 PM - 12:00 PM)}: 1L, sipped steadily until midnight  
\end{itemize}

\subsection*{Smoothie Composition}\vspace{-1.2em}
To ensure a nutrient-rich start to the day, I used the following standardized recipe:
\begin{itemize}
    \item \textbf{1 cup (150g) blueberries}: Antioxidants to reduce oxidative stress.
    \item \textbf{1 cup (150g) strawberries}: High in vitamin C to support immunity.
    \item \textbf{1 medium (120g) banana}: A source of potassium for muscle and heart health.
    \item \textbf{1 cup (30g) spinach}: Provides iron and other essential micronutrients.
    \item \textbf{1 tbsp (15g) flax seeds}: High in fiber and omega-3 fatty acids.
    \item \textbf{10g almonds and 10g walnuts}: Protein and healthy fats for sustained energy.
    \item \textbf{1 cup (240ml) almond milk}: Calcium and vitamin D.
\end{itemize}
Each smoothie contained approximately 350 calories, with 45g carbohydrates, 15g protein, and 12g fat.

\section*{Challenges and Solutions}\vspace{-1em}
I encountered several challenges but developed solutions for each. My morning smoothie preparation initially took too much time, but pre-portioning my ingredients in freezer bags reduced my preparation time by 10-15 minutes. Adapting to drinking 3.2L of water daily overwhelmed me at first, though setting my hourly reminders and flavoring my water with citrus or mint made my goal more achievable.

\section*{Observations, Outcomes and Nutritional Insights}\vspace{-1em}
After six weeks, I experienced several positive changes in my health. My digestion improved from the increased fiber in my smoothies, and I maintained steady energy levels throughout my day. My new morning routine enhanced my daily focus and productivity.

Through this challenge, I learned valuable lessons about nutrition. A well-balanced breakfast improved my energy and mood significantly, while consistent water intake enhanced my physical and cognitive performance.

\section*{Recommendations for Future Success}\vspace{-1em}
From my experience, I recommend pre-portioning ingredients to save time and reduce decision fatigue. Using tracking tools and reminders helped maintain my consistency. I learned to adapt my goals based on daily circumstances and focus on progress rather than perfection, celebrating small achievements to stay motivated.

\end{document}
