\documentclass[a4paper,12pt]{article}
\usepackage{amsmath}
\usepackage{amssymb}
\usepackage{geometry}
\geometry{margin=1in}
\usepackage{hyperref}

\title{Routh-Hurwitz Stability Criterion}
\date{}

\begin{document}

\maketitle

\section{Introduction}

In control systems, stability is of paramount importance. The Routh-Hurwitz criterion is a mathematical test that provides a way to determine the stability of a linear time-invariant (LTI) system without solving the characteristic equation directly. The criterion is based on the location of poles of the system's transfer function in the complex plane. Specifically, it tells us how many poles are in the left half-plane (stable region), the right half-plane (unstable region), and on the imaginary axis (marginally stable).\\

The characteristic equation of an LTI system is typically expressed as:
\begin{equation}
T(s) = a_0 s^n + a_1 s^{n-1} + a_2 s^{n-2} + \cdots + a_n = 0
\end{equation}
where \(a_0, a_1, \ldots, a_n\) are real coefficients, and \(s\) represents the complex variable.\\

The Routh-Hurwitz criterion uses a tabular method called the \textit{Routh array} to determine the number of roots with positive real parts (which correspond to unstable poles). If any poles have positive real parts, the system is unstable.

\section{Steps to Construct the Routh Array}

To apply the Routh-Hurwitz criterion, follow these steps:

\begin{enumerate}
    \item Write the characteristic equation in the standard form, where the highest power of \(s\) is first, followed by the descending powers.
    \item Create the first two rows of the Routh array using the coefficients of the characteristic equation.
    \item Construct the rest of the Routh array using a specific recursive formula, which involves taking the determinant of two-by-two submatrices of the array.
    \item Evaluate the first column of the Routh array. If all entries in the first column have the same sign, the system is stable. A change in sign indicates instability, and the number of sign changes corresponds to the number of poles in the right half-plane.
\end{enumerate}

\section{The Recursive Formula for the Routh Array}

Given the first two rows of the Routh array, the subsequent rows are computed as follows. Let the first two rows be:

\[
\begin{array}{c|ccccc}
s^n & a_0 & a_2 & a_4 & a_6 & \cdots \\
s^{n-1} & a_1 & a_3 & a_5 & a_7 & \cdots \\
\end{array}
\]

For a row \(s^k\), the elements are calculated using the formula:
\begin{equation}
b_i = \frac{a_1 \cdot a_{i+2} - a_0 \cdot a_{i+3}}{a_1}
\end{equation}
where \(a_0\) and \(a_1\) are the first elements of the two rows above the current row. This process continues for each subsequent row.

\section{Example 1: No Sign Changes}

Consider the characteristic equation:
\begin{equation}
T(s) = s^3 + 6s^2 + 11s + 6 = 0
\end{equation}

The first two rows of the Routh array are constructed using the coefficients of the equation:
\[
\begin{array}{c|ccc}
s^3 & 1 & 11 & \\
s^2 & 6 & 6 & \\
\end{array}
\]

Next, calculate the third row:
\[
b_1 = \frac{6 \cdot 11 - 1 \cdot 6}{6} = \frac{66 - 6}{6} = 10
\]
\[
\begin{array}{c|ccc}
s^3 & 1 & 11 & \\
s^2 & 6 & 6 & \\
s^1 & 10 & 0 & \\
s^0 & 6 & & \\
\end{array}
\]

In this case, all the elements in the first column are positive, indicating no sign changes. Hence, the system is \textbf{stable}.

\section{Example 2: Sign Changes (Unstable System)}

Consider the characteristic equation:
\begin{equation}
T(s) = s^3 - 2s^2 + 3s - 4 = 0
\end{equation}

The first two rows of the Routh array are:
\[
\begin{array}{c|ccc}
s^3 & 1 & 3 & \\
s^2 & -2 & -4 & \\
\end{array}
\]

Next, calculate the third row:
\[
b_1 = \frac{-2 \cdot 3 - 1 \cdot (-4)}{-2} = 1
\]
\[
\begin{array}{c|ccc}
s^3 & 1 & 3 & \\
s^2 & -2 & -4 & \\
s^1 & 1 & 0 & \\
s^0 & -4 & & \\
\end{array}
\]

Here, the first column contains alternating signs (1, -2, 1, -4), indicating \textbf{three sign changes}, which means the system has three poles in the right half-plane and is \textbf{unstable}.

\section{Example 3: Stable System with Higher Order}

Consider the fourth-order characteristic equation:
$$
T(s) = s^4 + 4s^3 + 2s^2 + 4s + 1 = 0
$$

To determine stability, let's construct the Routh array:

1) First row ($s^4$): Coefficients of $s^4$, $s^2$, $s^0$
$$
1 \quad 2 \quad 1
$$

2) Second row ($s^3$): Coefficients of $s^3$, $s^1$
$$
4 \quad 4 \quad 0
$$

3) Remaining rows are calculated using the formula:
$$
-\frac{1}{b_1}
\begin{vmatrix}
b_1 & b_2 \\
a_1 & a_2
\end{vmatrix}
$$

For $s^2$ row:
$$
b_1 = 4
$$
$$
\begin{vmatrix}
4 & 4 \\
1 & 2
\end{vmatrix} = 4(2) - 4(1) = 4
$$
$$
-\frac{1}{4}(4) = 1
$$

Similar calculations for remaining elements yield:

$$
\begin{array}{c|cccc}
s^4 & 1 & 2 & 1 & \\
s^3 & 4 & 4 & 0 & \\
s^2 & 1 & 1 & & \\
s^1 & 3 & 0 & & \\
s^0 & 1 & & & \\
\end{array}
$$

Stability Analysis:
1) The first column elements are: [1, 4, 1, 3, 1]
2) For stability, all elements in the first column must be positive
3) No sign changes are observed in the first column
4) Therefore, all roots (poles) of the characteristic equation lie in the left half-plane
5) Conclusion: The system is stable

The presence of all positive numbers in the first column without any sign changes confirms system stability, as all roots have negative real parts.

\section{Conclusion}

The Routh-Hurwitz criterion is a powerful method to determine the stability of a system by analyzing the characteristic equation without needing to solve for the exact roots. By constructing the Routh array and counting the number of sign changes in the first column, we can easily determine how many poles lie in the right half of the complex plane, and hence, infer the stability of the system. 

In this document, we explored three different examples: a stable system, an unstable system with multiple right-half plane poles, and a higher-order stable system, each illustrating different aspects of the Routh-Hurwitz stability criterion.

\end{document}