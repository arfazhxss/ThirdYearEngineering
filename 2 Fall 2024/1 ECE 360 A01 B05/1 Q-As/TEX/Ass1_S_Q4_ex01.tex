
\documentclass{article}
\usepackage{amsmath}
\usepackage[a4paper, left=0.5in, right=0.5in, top=0.5in, bottom=1in]{geometry}
\begin{document}

\title{Solution to Differential Equation}
\date{}
\maketitle

\section*{The Question}
We are given the following differential equation:
\[
2\ddot{x} + 7\dot{x} + 3x = u(t),
\]
with initial conditions \(x(0) = 3\) and \(\dot{x}(0) = 0\), where \(u(t)\) is the unit step function.

\section*{Step 1: Apply Laplace Transform}

To solve this differential equation, we begin by applying the Laplace transform to both sides. The Laplace transform of a derivative is:

\[
\mathcal{L}\{\dot{x}(t)\} = sX(s) - x(0),
\]
\[
\mathcal{L}\{\ddot{x}(t)\} = s^2 X(s) - s x(0) - \dot{x}(0).
\]

Applying the Laplace transform to the entire equation:

\[
2\mathcal{L}\{\ddot{x}(t)\} + 7\mathcal{L}\{\dot{x}(t)\} + 3\mathcal{L}\{x(t)\} = \mathcal{L}\{u(t)\},
\]

where \(\mathcal{L}\{u(t)\} = \frac{1}{s}\).

Using the initial conditions \(x(0) = 3\) and \(\dot{x}(0) = 0\), we substitute into the transformed equation:

\[
2(s^2 X(s) - 3s) + 7(s X(s) - 3) + 3X(s) = \frac{1}{s}.
\]

\section*{Step 2: Simplify the Equation}

Now, simplify the Laplace-transformed equation:

\[
2(s^2 X(s) - 3s) + 7(s X(s) - 3) + 3X(s) = \frac{1}{s}.
\]

Expanding both terms:

\[
2s^2 X(s) - 6s + 7sX(s) - 21 + 3X(s) = \frac{1}{s}.
\]

Collect all terms involving \(X(s)\) on the left-hand side:

\[
(2s^2 + 7s + 3)X(s) = \frac{1}{s} + 6s + 21.
\]

\section*{Step 3: Solve for \(X(s)\)}

We now solve for \(X(s)\):

\[
X(s) = \frac{\frac{1}{s} + 6s + 21}{2s^2 + 7s + 3}.
\]

Multiply the numerator by \(s\) to clear the fraction:

\[
X(s) = \frac{1 + 6s^2 + 21s}{s(2s^2 + 7s + 3)}.
\]

\section*{Step 4: Partial Fraction Decomposition}

To proceed, we perform partial fraction decomposition on \(X(s)\). The poles of the denominator are obtained by factoring \(2s^2 + 7s + 3\). The roots of the quadratic are found using the quadratic formula:

\[
s = \frac{-7 \pm \sqrt{49 - 4(2)(3)}}{2(2)} = \frac{-7 \pm 1}{4}.
\]

So the poles are \(s = -\frac{1}{2}\) and \(s = -3\). Thus, we can express \(X(s)\) as:

\[
X(s) = \frac{A}{s} + \frac{B}{s + \frac{1}{2}} + \frac{C}{s + 3}.
\]

\section*{Step 5: Solve for Constants \(A\), \(B\), and \(C\)}

To find \(A\), \(B\), and \(C\), we multiply both sides of the equation by \(s(s + \frac{1}{2})(s + 3)\) and match coefficients. After solving, we get the values:

\[
A = \frac{1}{3}, \quad B = \frac{16}{5}, \quad C = -\frac{8}{15}.
\]

Thus, we can write:

\[
X(s) = \frac{1}{3s} + \frac{16}{5(s + \frac{1}{2})} - \frac{8}{15(s + 3)}.
\]

\section*{Step 6: Apply Inverse Laplace Transform}

Now, apply the inverse Laplace transform to each term separately:

\[
\mathcal{L}^{-1}\left\{\frac{1}{3s}\right\} = \frac{1}{3}u(t),
\]
\[
\mathcal{L}^{-1}\left\{\frac{16}{5(s + \frac{1}{2})}\right\} = \frac{16}{5} e^{-\frac{1}{2}t},
\]
\[
\mathcal{L}^{-1}\left\{\frac{8}{15(s + 3)}\right\} = -\frac{8}{15}e^{-3t}.
\]

Thus, the solution for \(x(t)\) is:

\[
x(t) = \frac{1}{3}u(t) + \frac{16}{5}e^{-\frac{1}{2}t} - \frac{8}{15}e^{-3t}, \quad t \geq 0.
\]

\section*{Final Answer}

The solution to the differential equation is:

\[
x(t) = \frac{1}{3}u(t) + \frac{16}{5}e^{-\frac{1}{2}t} - \frac{8}{15}e^{-3t}, \quad t \geq 0.
\]

\end{document}
