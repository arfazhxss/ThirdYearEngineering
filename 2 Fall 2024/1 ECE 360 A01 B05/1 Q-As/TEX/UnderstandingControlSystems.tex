\documentclass{article}
\usepackage{amsmath}
\usepackage{graphicx}
\usepackage{tikz}
\usepackage{float}

\title{Understanding Basic Control Systems}
\author{Control Systems Guide}
\date{}

\begin{document}
\maketitle

\section{Introduction to Feedback Control Systems}
A basic unity feedback control system consists of the following components:
\begin{itemize}
    \item Input signal R(s): The desired output or reference signal
    \item Error signal E(s): The difference between desired and actual output
    \item Controller gain K: Amplification factor
    \item Plant/Process G(s): The system being controlled
    \item Output signal C(s): The actual system output
\end{itemize}

\section{Key Terminology}

\subsection{System Response Characteristics}
\begin{itemize}
    \item \textbf{Step Response}: The system's output behavior when the input changes instantaneously from zero to a constant value
    \item \textbf{Rise Time}: Time taken for the output to go from 10\% to 90\% of its final value
    \item \textbf{Settling Time}: Time taken for the system to reach and stay within ±2\% of its final value
    \item \textbf{Overshoot}: Maximum amount the system exceeds its final steady-state value, expressed as a percentage
    \item \textbf{Steady-state Error}: The difference between desired and actual output when time approaches infinity
\end{itemize}

\subsection{Damping Characteristics}
\begin{itemize}
    \item \textbf{Overdamped}: System returns to steady state without oscillating
        \begin{itemize}
            \item Slower response
            \item No overshoot
            \item Typically occurs with small K values
        \end{itemize}
    \item \textbf{Underdamped}: System oscillates before reaching steady state
        \begin{itemize}
            \item Faster initial response
            \item Has overshoot
            \item Multiple oscillations
            \item Common with larger K values
        \end{itemize}
    \item \textbf{Critically Damped}: Fastest response without overshoot
        \begin{itemize}
            \item Optimal balance between speed and stability
            \item Occurs at specific K value
        \end{itemize}
\end{itemize}

\section{Effect of Gain (K) on System Response}

\subsection{K = 0}
\begin{itemize}
    \item System is completely unresponsive
    \item Output C(s) = 0 for any input
    \item Equivalent to open-loop system
\end{itemize}

\subsection{Small K Values}
\begin{itemize}
    \item Slow, stable response
    \item Typically overdamped
    \item Large steady-state error
    \item No overshoot
\end{itemize}

\subsection{Medium K Values}
\begin{itemize}
    \item Faster response
    \item May become underdamped
    \item Moderate overshoot
    \item Reduced steady-state error
\end{itemize}

\subsection{Large K Values}
\begin{itemize}
    \item Very fast initial response
    \item Significant overshoot
    \item Multiple oscillations
    \item Small steady-state error
    \item Longer settling time due to oscillations
\end{itemize}

\subsection{K → ∞}
\begin{itemize}
    \item Extremely oscillatory behavior
    \item May become unstable
    \item Theoretically zero steady-state error
    \item Not practically useful
\end{itemize}

\section{Mathematical Representation}
For a unity feedback system:
\begin{align*}
    E(s) &= R(s) - C(s) \\
    C(s) &= KG(s)E(s) \\
    \text{Transfer Function } T(s) &= \frac{C(s)}{R(s)} = \frac{KG(s)}{1 + KG(s)}
\end{align*}

\section{Universal Principles}
Regardless of the specific G(s):
\begin{itemize}
    \item Higher K generally means faster initial response
    \item Higher K leads to more oscillatory behavior
    \item There's always a trade-off between speed and stability
    \item Steady-state error generally decreases as K increases
\end{itemize}

The main differences between systems with different G(s) are:
\begin{itemize}
    \item The K value at which underdamped behavior begins
    \item Whether there's a K value that causes instability
    \item The specific pattern of oscillations
    \item The rate at which the response characteristics change with K
\end{itemize}

\end{document}