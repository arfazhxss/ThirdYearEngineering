\documentclass{article}
\usepackage{amsmath}
\usepackage{array}
\usepackage{booktabs}
\usepackage{multirow}
\usepackage[margin=1.4in]{geometry}
\usepackage{xcolor}
\usepackage{listings}
\usepackage{textcomp}

% Define custom colors for assembly code
\definecolor{bgcolor}{rgb}{0.95,0.95,0.92}
\definecolor{opcolor}{rgb}{0.0,0.0,0.6}
\definecolor{registercolor}{rgb}{0.5,0,0.5}
\definecolor{commentcolor}{rgb}{0.4,0.4,0.4}
\definecolor{immediatecolor}{rgb}{0.0,0.5,0.0}
\definecolor{linenumbergray}{rgb}{0.3,0.3,0.3}

% Custom listing style for assembly
\lstdefinestyle{customasm}{
    language=[x86masm]Assembler,
    backgroundcolor=\color{bgcolor},
    basicstyle=\ttfamily\small,
    numbers=left,
    numberstyle=\color{linenumbergray}\ttfamily\small,
    stepnumber=1,
    numbersep=10pt,
    frame=single,
    rulecolor=\color{black},
    tabsize=4,
    captionpos=b,
    breaklines=true,
    breakatwhitespace=false,
    commentstyle=\color{commentcolor}\itshape,
    keywordstyle=\color{opcolor},
    keywordstyle=[2]\color{registercolor},
    stringstyle=\color{immediatecolor},
    showspaces=false,
    showstringspaces=false,
    showtabs=false,
    escapeinside={\%*}{*)},
    morekeywords={MOV,ADD,SUB,NOP},
    morekeywords=[2]{R0,R1,R2,R3,R4,R5},
    morecomment=[l]{;}
}

% Define colors for C code highlighting
\definecolor{ccommentcolor}{rgb}{0.4,0.4,0.4}
\definecolor{ckeywordcolor}{rgb}{0.0,0.0,0.6}
\definecolor{cstringcolor}{rgb}{0.0,0.5,0.0}
\definecolor{cpreprocessorcolor}{rgb}{0.6,0.0,0.0}
\definecolor{cnumbercolor}{rgb}{0.0,0.5,0.5}

% Custom listing style for C code
\lstdefinestyle{customc}{
    language=C,
    backgroundcolor=\color{bgcolor},
    basicstyle=\ttfamily\small,
    numbers=left,
    numberstyle=\color{linenumbergray}\ttfamily\small,
    stepnumber=1,
    numbersep=10pt,
    frame=single,
    rulecolor=\color{black},
    tabsize=4,
    captionpos=b,
    breaklines=true,
    breakatwhitespace=false,
    % Styling for different elements
    commentstyle=\color{ccommentcolor}\itshape,
    keywordstyle=\color{ckeywordcolor}\bfseries,
    stringstyle=\color{cstringcolor},
    numberstyle=\color{linenumbergray}\ttfamily\small,
    % Preprocessor styling
    identifierstyle=\color{black},
	moredelim=[is][\color{cpreprocessorcolor}]{@}{@},
    % Other settings
    showspaces=false,
    showstringspaces=false,
    showtabs=false,
    % Special keywords for C
    morekeywords={inline,int,long,char,float,double,bool,void,unsigned,signed,
                  const,static,struct,union,enum,typedef,volatile,register,
                  extern,restrict,sizeof,_Bool,_Complex,_Imaginary,
                  if,else,for,while,do,switch,case,default,break,continue,return,
                  goto},
    % Extra settings for C-specific features
    morecomment=[l]{//},
    morecomment=[s]{/*}{*/},
    morestring=[b]",
    morestring=[b]',
    % Support for numbers
	literate=
    	{0}{{{\color{cnumbercolor}0}}}{1}
    	{1}{{{\color{cnumbercolor}1}}}{1}
    	{2}{{{\color{cnumbercolor}2}}}{1}
    	{3}{{{\color{cnumbercolor}3}}}{1}
    	{4}{{{\color{cnumbercolor}4}}}{1}
    	{5}{{{\color{cnumbercolor}5}}}{1}
    	{6}{{{\color{cnumbercolor}6}}}{1}
    	{7}{{{\color{cnumbercolor}7}}}{1}
    	{8}{{{\color{cnumbercolor}8}}}{1}
    	{9}{{{\color{cnumbercolor}9}}}{1}
}

% Set the default listing style
\lstset{style=customasm}

\begin{document}

\begin{center}
    \textbf{\LARGE Assignment 6} \\[1ex]
    \textbf{ECE 355} \\[1ex]
    \textbf{V00984826} \\[2ex]
\end{center}

\section*{Problem 1}

Answer the following questions about IEEE-754 32-bit floating-point representation:

\begin{enumerate}
    \item[1.] (10 points)
    \begin{enumerate}
        \item[(a)] Convert the IEEE-754 number \(1 \; 11111111 \; 00000000000000000000000\) to its decimal format.
        \item[(b)] Convert the IEEE-754 number \(0 \; 00000000 \; 11000000000000000000000\) to its decimal format.
        \item[(c)] Represent the decimal number \(-0.625\) in the 32-bit IEEE-754 format.
        \item[(d)] Given the two IEEE-754 numbers \(X\) and \(Y\) below:
        \[
        X = 0 \; 01111011 \; 10000000000000000000000, \quad
        Y = 1 \; 01111110 \; 11010000000000000000000,
        \]
        calculate \(Z = X - Y\) in binary format. Then, convert the resulting \(Z\) from IEEE-754 binary format to its decimal equivalent.
    \end{enumerate}
\end{enumerate}

\subsection*{Solution 1}

\subsubsection*{Part a)}
For \(1 \; 11111111 \; 00000000000000000000000\):
\begin{align*}
M_1 &= 0 \\
E &= 255 \\
\text{Result} &= \text{Not a number (NaN)}
\end{align*}

\subsubsection*{Part b)}
For \(0 \; 00000000 \; 11000000000000000000000\):
\begin{align*}
S &= 0 \\
E &= 0 \\
M &= 1.1 \rightarrow 0 \\
E &= 0 \\
\text{Number} &= +1 \times 2^{-126} \times 0.11 \\
&= 0.75 \times 2^{-126} \\
&= 8.8162076 \times 10^{-39}
\end{align*}

\subsubsection*{Part c)}
For \(-0.625\):
\begin{align*}
0.625 &= 0.5 + 0.125 \\
&= 2^{-1} + 2^{-3} \\
&= (0.101)_2 \times 2^0 \\
&= (1.01)_2 \times 2^{-1} \\
\text{Sign} &= 1 \\
\text{Exponent} &= -1 + 127 = 126 = (01111110)_2 \\
\text{Significand} &= 01000000000000000000000 \\
\text{IEEE-754} &= 1 \; 01111110 \; 01000000000000000000000
\end{align*}

\subsubsection*{Part d)}
Given two 32-bit IEEE-754 numbers:

$$X = \text{0 01111011 10000000000000000000000}$$
$$Y = \text{1 01111110 11010000000000000000000}$$

For X:
$$X = 1.5 \times 2^{-4} = 0.09375$$

For Y: 
$$Y = -1.8125 \times 2^{-1} = -0.90625$$

Therefore:
$$Z = X - Y = 0.09375 - (-0.90625) = 1.0$$

Converting to IEEE-754:
$$Z = \text{0 01111111 00000000000000000000000}$$

\section*{Problem 2}

(5 points) Consider a pipelined datapath consisting of five stages:
\begin{itemize}
    \item \textbf{F} -- Fetch the instruction from the memory,
    \item \textbf{D} -- Decode the instruction and read the source register(s),
    \item \textbf{C} -- Execute the ALU operation specified by the instruction,
    \item \textbf{M} -- Execute the memory operation specified by the instruction,
    \item \textbf{W} -- Write the result in the destination register.
\end{itemize}

\text{\\\\Identify data hazards in the code below and insert \textbf{NOP} instructions where necessary.}

\begin{lstlisting}
ADD #4, R0, R0         ; R0 = R0 + 4
ADD #4, R2, R2         ; R2 = R2 + 4
MOV (R0), R1           ; R1 = MEMORY[R0]
MOV (R2), R3           ; R3 = MEMORY[R2]
SUB R2, R0, R4         ; R4 = R2 – R0
SUB R3, R1, R5         ; R5 = R3 – R1
MOV R4, (R2)           ; MEMORY[R2] = R4
MOV R5, (R0)           ; MEMORY[R0] = R5
ADD #4, R0, R0         ; R0 = R0 + 4
ADD #4, R2, R2         ; R2 = R2 + 4
\end{lstlisting}

\subsection*{Solution 2}
To resolve the data hazards in the given code, \textbf{NOP} (no-operation) instructions are inserted where necessary. The modified code with \textbf{NOP} instructions is as follows:

\begin{lstlisting}
ADD #4, R0, R0        ; R0 = R0 + 4
ADD #4, R2, R2        ; R2 = R2 + 4
NOP
NOP
MOV (R0), R1          ; R1 = MEMORY[R0]
MOV (R2), R3          ; R3 = MEMORY[R2]
SUB R2, R0, R4        ; R4 = R2 - R0
NOP
NOP
SUB R3, R1, R5        ; R5 = R3 - R1
MOV R4, (R2)          ; MEMORY[R2] = R4
NOP
NOP
MOV R5, (R0)      	  ; MEMORY[R0] = R5
NOP
NOP
NOP
ADD #4, R0, R0        ; R0 = R0 + 4
ADD #4, R2, R2        ; R2 = R2 + 4
\end{lstlisting}

\textbf{\\Explanation:}
\begin{itemize}
    \item After the first two \texttt{ADD} instructions, \textbf{NOP} instructions are inserted to prevent data hazards caused by dependent instructions that use \texttt{R0} and \texttt{R2}.
    \item For the \texttt{SUB} instructions, additional \textbf{NOP} instructions are added because the values of \texttt{R0} and \texttt{R2} (updated by \texttt{ADD} or \texttt{MOV}) are required before proceeding.
    \item Similarly, after the \texttt{MOV} instructions that store the results in memory, \textbf{NOP} instructions are added to allow time for memory updates.
    \item Additional \textbf{NOP} instructions are inserted before the final \texttt{ADD} instructions to ensure no unresolved hazards.
\end{itemize}

\section*{Problems 3 and 4}

\begin{enumerate}
    \item[3.] (2 points) Solve Problem 12.8 from the textbook.
    \item[4.] (8 points) Solve Problem 12.7 from the textbook. 

    \textbf{Hint:} Declare a shared counter variable, e.g., \texttt{volatile int thread\_id\_counter}, initialize it to 0 in \texttt{main()}, and poll it by each thread as follows: 
    \[
    \texttt{while (thread\_id\_counter != my\_id);}
    \]
    Each thread must increment \texttt{thread\_id\_counter} after updating \texttt{global dot\_product}.
\end{enumerate}

\subsection*{Solution for Problem 3}

Given \( P = 8 \) and \( \text{Speedup} = 5 \), we need to solve the equation:
\[
5 = \frac{1}{1 - f + \frac{f}{8}}
\]
Rewriting the equation:
\[
1 - f + \frac{f}{8} = \frac{1}{5}
\]
\[
1 - f + \frac{f}{8} = 0.2
\]
\[
1 - 0.2 = f - \frac{f}{8}
\]
\[
0.8 = f \left(1 - \frac{1}{8}\right)
\]
\[
0.8 = f \cdot \frac{7}{8}
\]
\[
f = \frac{0.8 \cdot 8}{7} = 0.914
\]

Thus, \( f = 0.914 \), meaning the application program must be \( 91\% \) parallelizable.

\subsection*{Solution for Problem 4}

Below is the code for solving the problem using threads and a shared counter for synchronization. The shared variable ensures that each thread updates the global \texttt{dot\_product} in a coordinated manner and avoiding race conditions.

\lstset{style=customc}

\begin{lstlisting}
#include <stdio.h>
#include "threads.h"

#define N 100
#define P 4

double a[N], b[N];
double dot_product = 0.0;  // Global variable for storing the result
volatile int thread_id_counter = 0;  // Shared counter for synchronization
Barrier bar;

void ParallelFunction (void)
{
    int my_id, i, start, end;
    double s;
    my_id = get_my_thread_id();
    start = (N/P) * my_id;
    end = (N/P) * (my_id + 1) - 1;

    // Step 1: Compute partial dot product for the assigned range
    s = 0.0;
    for (i = start; i <= end; i++)
        s = s + a[i] * b[i];

    // Step 2: Synchronize threads to update global sum one at a time
    while (thread_id_counter != my_id); // Wait for the current thread's turn
    dot_product = dot_product + s;     // Update the global dot product
    thread_id_counter = (thread_id_counter + 1) % P; // Allow the next thread access

    // Step 3: Synchronize all threads using a barrier
    barrier(&bar, P);
}

void main(void)
{
    int i;

    /* Initialize vectors a[] and b[] (details omitted for brevity). */
    init_barrier(&bar);                // Initialize the barrier for thread synchronization
    thread_id_counter = 0;             // Initialize the shared counter
    dot_product = 0.0;                 // Initialize the global dot product

    // Step 4: Create threads for parallel computation
    for (i = 1; i < P; i++)
        create_thread(ParallelFunction);
    ParallelFunction();                // Execute the function in the main thread

    // Print the final result
    printf("The dot product is %g\n", dot_product);
}
\end{lstlisting}

\textbf{\\Explanation:}
\begin{itemize}
    \item \textbf{Global variables:} The \texttt{dot\_product} variable is used to store the final result of the dot product calculation. The \texttt{thread\_id\_counter} ensures that threads update the global variable sequentially, avoiding race conditions.
    \item \textbf{Synchronization:} The \texttt{while} loop synchronizes threads by making them wait for their turn to update the global \texttt{dot\_product}. The modulo operation ensures that access rotates among the threads.
    \item \textbf{Barrier:} After updating the global variable, threads wait at a barrier to ensure all threads complete their computations before proceeding.
    \item \textbf{Thread creation:} Threads are created in a loop, and each thread computes a portion of the dot product based on its assigned range.
    \item \textbf{Main thread:} The main thread also participates in the computation by executing \texttt{ParallelFunction()}.
\end{itemize}

This implementation ensures correct and efficient parallel computation of the dot product.

\end{document}